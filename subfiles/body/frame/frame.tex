%%%%%%%%%%%%%%%%%%%%%%%%%%%
\section{フレーム}
%%%%%%%%%%%%%%%%%%%%%%%%%%%
フレームでは,全体目標であるエンデュランス1320 secの達成の為に車両応答性の向上に加え,コクピット班の目標であるComfortabilityの向上を目標として設計を行った.

まず設計を行う前に,フレームが車両挙動に及ぼす影響を調べるためにKS-14のねじり剛性,横曲げ剛性を変化させて舵角,横G,右側後輪のダンパーのストローク量を模擬コースの走行により測定を行った.この結果,最も横Gが大きかったスラローム区間においてデータを比較したところ,Fig.\ref{fig:frame1}に示すように右側の後輪に取り付けたストロークセンサーから横Gに対する変位量は剛性に対してほぼ変化が無いという結果が得られた.次に操舵に対する横Gの位相を調べたところ,剛性が増加するにつれて位相の遅れが小さくなっていることが確認できた.この応答性は横曲げ剛性に起因すると考え,’18年度はフレームの横曲げ剛性を向上させることにより,操舵に対する横Gの位相遅れを小さくすることを目標にした.

%%%%%%%%%%%%%%%%%%%%%%%%%%%
\subsection{設計概要}
%%%%%%%%%%%%%%%%%%%%%%%%%%%
上述の目標から,KS-15ではKS-14からフレームの重量を変化させず,その中で可能な限り横曲げ剛性を向上させるように設計を行った.その結果’18年度の重量と横曲げ剛性はCATIA上でそれぞれ31.2 kg,16×10$^5$ N-mとなり,前年度の31.1 kg,9×10$^5$ N-mから重量をほぼ変化させずに横曲げ剛性を7×10$^5$ N-m向上させることが出来た.また,設計通りの性能を発揮させるためには,旋回時のサスペンションの動きを設計に忠実に動作させる必要があると考えた.そこで,Aアームやトーロッドのフレーム取り付け部の変位を抑えるため,前年度と同様に取り付け部を可能な限りフレームのノードに取り付けられるように設計を行うことに加え,取り付けられなかった箇所については小径パイプにより変位を抑えた.

次にComfortabilityの向上について,シートと共同で設計を進めドライバーの姿勢を決定した後にフレームのレイアウトの設計を行った.これにより,全年度フレームに比べてシートステイロア部分の幅を330 mmから430 mmに増加させシートの幅を十分に取れるように変更した.また,Fr.フープの上部に新たに曲げを入れることによりステアリングの位置を上げつつドライバーの視界を確保することができた.
