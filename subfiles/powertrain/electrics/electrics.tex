%%%%%%%%%%%%%%%%%%%%%%%%%%%
\subsection{電装}
%%%%%%%%%%%%%%%%%%%%%%%%%%%
KS-15では吸排気の設計時に得た解析ソフト上のパワー特性に近づけることを目標に燃調セッティングを行った.その際,より自由度の高いセッティングを行うために,KS-14に引き続きMoTeCを使用した.具体的には,エンジンを回しながらA/Fセンサーの値が理想空燃比に近づくようにセッティングを進めていき,ダイナパックで計測したパワー特性を見て修正を加えるという方法をとった.

アクセラレーションのタイム向上を目指し,ローンチコントロール実装によって発進時のタイヤの空転を抑える.これを達成するために,フロント・リアの回転部にスピードセンサを取り付けて前後輪の回転数を取り,後輪が前輪の110 \%の回転数になった時に燃料カットによってローンチコントロールが作動するようにした.

今後は,この手法をトラクションコントロールの実装に応用し,コーナリングスピードアップを目指す.
