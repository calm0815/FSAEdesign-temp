%%%%%%%%%%%%%%%%%%%%%%%%%%%
\subsection{エンジン}
%%%%%%%%%%%%%%%%%%%%%%%%%%%
エンジンは,パワートレインの開発目標からメインストレートでの加速性能や最高速を重視した.したがって使用するエンジンは単気筒よりも出力の大きい4気筒エンジンとした.また昨年度との性能比較が容易といった観点から,KS-14と同様Kawasaki ZX-600PEとした。

%%%%%%%%%%%%%%%%%%%%%%%%%%%
\subsection{バルブタイミング}
%%%%%%%%%%%%%%%%%%%%%%%%%%%
パワートレインの開発目標を達成するため,常用回転域における最大トルク及び最高出力の向上を目標にバルブタイミングの再検討を行った.

KS-14では低回転域でのエンジン応答性向上やアイドリング安定による騒音の低減,再始動性向上を目的とし吸気側バルブタイミングを標準より22.5$^\circ$遅角させていた.しかし低回転域でのトルクや最高出力がダイナパック測定結果より悪化することが分かっていた.したがって,2017年度大会後にKS-14のエンジンモデルをGT-Power上で作成し標準バルブタイミングとの比較を行った.また出力と上述した目的の両立を狙い吸気側を14.4$^\circ$遅角及び排気側を14.4$^\circ$進角したものの検討も行った.Fig.\ref{fig:power1}がその結果である.これより22.5$^\circ$遅角すると最高出力発生点付近(14000 rpm)ではそこまで出力に差はないが,常用回転域全域でみると大幅に悪化することが分かった.次に吸気側を14.4$^\circ$遅角及び排気側を14.4$^\circ$進角したものについてもそこまでトルクの落ち込みを抑えることが出来なかった.ドライバーからのフィードバックやアクセラレーションの結果比較より,実走行においても応答性やタイムに望ましい結果が得られなかった.

アイドリングの安定や再始動性向上についてはMoTeCの本格導入やサイレンサの設計変更により対応可能であることから,今年度は常用回転域での出力を重視しバルブタイミングを標準とすることとした.
