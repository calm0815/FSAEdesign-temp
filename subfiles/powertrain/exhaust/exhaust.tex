%%%%%%%%%%%%%%%%%%%%%%%%%%%
\subsection{排気}
%%%%%%%%%%%%%%%%%%%%%%%%%%%
排気系設計ではパワートレインの目標を達成するため常用回転域での更なるトルク向上と扱いやすい出力特性を目指し開発を行った.目標として常用回転域での最大軸トルクを60 N-mとした(前年度45N-m).加えて,ドライバーが扱いやすいリニアなトルク特性となるようにエンジンモデルを選定した.

またKS-14が孕んでいた様々な問題についても解決策を模索した.

%%%%%%%%%%%%%%%%%%%%%%%%%%%
\subsubsection{排気管}
%%%%%%%%%%%%%%%%%%%%%%%%%%%
まずKS-14の問題点として解析方法の未確立及び前年度との比較不足が挙げられた.これは,9月の段階にはGT-powerでKS-14のエンジンモデルを製作して解析を行い、比較対象を明確化することで解決した.次にKS-15では排気管の軽量化にも取り組み,排気系パーツ一つ一つの重量を実測・管理を徹底した.またKS-14から採用したはめ込み式の排気管は取り付けがしづらかったため,はめ込み式は踏襲しつつ集合部のみを一体形状とし整備性の向上に努めた.集合部は左右対称形状とすることで排気管の取り回し設計や製作を行いやすくした.またA/Fセンサーマウントを各気筒に配置することで,気筒別に燃調マッピングが行える構造とした.(Fig.\ref{fig:power3})

次に軸トルク向上及び扱いやすい出力特性を達成するため,吸気側が先に製作した’17年度エンジンモデルを参考に,主に排気管長と管径を変更しながら10個ほどのモデルを作成した.これらの解析結果から,管径よりも管長,特に集合部よりも前の管長(プライマリ)が出力特性に大きく影響を及ぼすことが分かった.そこで集合部以下を燃料タンクやフレームとの干渉を考慮しつつ必要最小限の管長・管径とした.管径については解析結果から僅かな差であるが$\phi$38.1が最適であったため,それ以降の解析は管長のみを変更し管径は常に$\phi$38.1で行った.以上の方法で約40個のモデルを作成し,その中から目標最大トルクを達成しかつリニアなトルク特性となるものを選定した.その結果プライマリ管長は495 mmとなった.Fig.\ref{fig:power4}にKS-14の標準バルブタイミングのものとKS-15のそれぞれのトルク・PS性能曲線を示す.Fig.\ref{fig:power4}から分かるように常用回転域において大幅にトルクが向上したことが分かる.また最大トルクは7500 rpmにおいて60.1 N-mとなり目標数値を達成した.また最小限の管長・管径としたことにより排気管で200 gの軽量化を実現することが出来た.

%%%%%%%%%%%%%%%%%%%%%%%%%%%
\subsubsection{サイレンサ}
%%%%%%%%%%%%%%%%%%%%%%%%%%%
KS-14においては,アイドリング時の騒音低減を狙ってチャンバー(サブサイレンサ)を搭載していた.しかしチャンバー単体で1200 gもあることに加えスペースを必要とするためサイレンサを外側に配置しなければならず慣性モーメントの増加を招いていた.そこでKS-15ではまずチャンバー廃止によるエンジン出力への影響を検討した.Fig.\ref{fig:power5}がGT-powerで解析した結果であるが,出力への影響は殆どなかったため廃止をすることに決定した.しかし騒音については全域において音圧レベルが増加することが解析結果から分かったため,増加分をサイレンサエンド部を延長することにより消音することとした.

サイレンサエンド部の長さを10 mmずつ長さの違うモデルを複数個作成しGT-powerを用いて評価・選定を行った.その結果,サイレンサエンド部が550 mmのときアイドリング時2500 rpmのとき100.2 dBC,11000 rpmのとき108.7 dBCとなり必要な消音性能を得ることが出来た(Fig.\ref{fig:power6}).
