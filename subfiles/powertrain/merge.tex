%%%%%%%%%%%%%%%%%%%%%%%%%%%
\section{パワートレイン}
%%%%%%%%%%%%%%%%%%%%%%%%%%%
チーム目標を達成するためパワートレインのベンチマークよりも劣っている点を’17年度大会のエンデュランス区間タイムから分析を行った.
1.1項の分析の結果,低回転域からの加速性能がベンチマークよりも劣っていることが分かった.そこでパワートレインでは,更にその区間での分析を行い,1コーナーからストレートエンド到達までのタイム差がベンチマークよりも平均して0.4 sec遅く,最もタイムに影響を及ぼしていることに着目した.

ベンチマークでは上述の区間を60 mとすると,50 mを2.5 secで通過していることが分かった.従ってKS-15でのパワートレイン開発目標を,1コーナー脱出速度をロガーデータ等から40 km/hとし,「メインストレート50 mを2.5 secで走行可能な加速性能」とし開発を行った. 
またベンチマークの調査結果から,最高出力は最低80 PSを維持し,コーナー脱出時(低回転域)からの常用回転域でのトルク特性を重視した.常用回転域はKS-14のロガーデータ及びオンボード映像から5000~13000 rpmとした.

具体的には,まず吸気がKS-14の解析モデルをもとに設計を行い,それが終了次第排気が設計を行うという手順で行った.そして実走行において燃調セッティングにより解析値に近づける手法をとった.

%エンジン
%%%%%%%%%%%%%%%%%%%%%%%%%%%
\subsection{エンジン}
%%%%%%%%%%%%%%%%%%%%%%%%%%%
エンジンは,パワートレインの開発目標からメインストレートでの加速性能や最高速を重視した.したがって使用するエンジンは単気筒よりも出力の大きい4気筒エンジンとした.また昨年度との性能比較が容易といった観点から,KS-14と同様Kawasaki ZX-600PEとした。

%%%%%%%%%%%%%%%%%%%%%%%%%%%
\subsection{バルブタイミング}
%%%%%%%%%%%%%%%%%%%%%%%%%%%
パワートレインの開発目標を達成するため,常用回転域における最大トルク及び最高出力の向上を目標にバルブタイミングの再検討を行った.

KS-14では低回転域でのエンジン応答性向上やアイドリング安定による騒音の低減,再始動性向上を目的とし吸気側バルブタイミングを標準より22.5$^\circ$遅角させていた.しかし低回転域でのトルクや最高出力がダイナパック測定結果より悪化することが分かっていた.したがって,2017年度大会後にKS-14のエンジンモデルをGT-Power上で作成し標準バルブタイミングとの比較を行った.また出力と上述した目的の両立を狙い吸気側を14.4$^\circ$遅角及び排気側を14.4$^\circ$進角したものの検討も行った.Fig.\ref{fig:power1}がその結果である.これより22.5$^\circ$遅角すると最高出力発生点付近(14000 rpm)ではそこまで出力に差はないが,常用回転域全域でみると大幅に悪化することが分かった.次に吸気側を14.4$^\circ$遅角及び排気側を14.4$^\circ$進角したものについてもそこまでトルクの落ち込みを抑えることが出来なかった.ドライバーからのフィードバックやアクセラレーションの結果比較より,実走行においても応答性やタイムに望ましい結果が得られなかった.

アイドリングの安定や再始動性向上についてはMoTeCの本格導入やサイレンサの設計変更により対応可能であることから,今年度は常用回転域での出力を重視しバルブタイミングを標準とすることとした.


%吸気
%%%%%%%%%%%%%%%%%%%%%%%%%%%
\subsection{吸気}
%%%%%%%%%%%%%%%%%%%%%%%%%%%
昨年度の吸気系はサージタンク容量の増加とそれに伴う利用可能スペースの減少により,応答性の悪化と出力特性の低下を招いた.そこで,今年度の吸気の目標を応答性の向上と常用回転域における出力特性,特にコーナー脱出時(5000〜8000 rpm)における特性の向上とした.

%%%%%%%%%%%%%%%%%%%%%%%%%%%
\subsubsection{スロットルボディ}
%%%%%%%%%%%%%%%%%%%%%%%%%%%
スロットル径の検討について昨年までの径$\phi$25と一昨年の径$\phi$32のリストリクターを含めたモデルを三次元熱流体解析ソフトSCRYU/Tetraで定常解析を行った.スロットル開度ごとの質量流量を算出した結果,$\phi$25ではスロットル開度の増加に伴い質量流量が比較的線形で変化しているが,$\phi$32では質量流量は全体的に増加したがスロットル開度が80を超えると質量流量が減少した(Fig.\ref{fig:intake1}).また,GT-powerを用いた解析では$\phi$25と$\phi$32で出力特性に差は見られなかった.そこで,今年度は$\phi$25と$\phi$32のスロットルを製作し実走行にて評価を行った.

%%%%%%%%%%%%%%%%%%%%%%%%%%%
\subsubsection{スロットル径}
%%%%%%%%%%%%%%%%%%%%%%%%%%%
今年はGT-powerの解析結果によってスロットルの径を実走行での結果を基に決定するため,$\phi$25と$\phi$32のスロットルを製作した.これらのスロットルの形状は,ANSYS CFXでの解析結果を基に決定した.質量流量の向上を目標に設計する事により質量流量を昨年より最大で3.8 \%向上させることに成功した.

%%%%%%%%%%%%%%%%%%%%%%%%%%%
\subsubsection{出力特性}
%%%%%%%%%%%%%%%%%%%%%%%%%%%
GT-powerを用いて吸排気系でのパワー・トルク特性の評価を行った.吸排気での設計は吸気系がマシンの出力特性に及ぼす影響が大きいので吸気系の設計を先に行った.GT-powerでの設計は昨年度モデルからバルブタイミングの変更と吸気系のパラメータを変更し,主に吸気管長とタンク容量の決定を行った.ここで,吸気管長はファンネルから吸気ポートまでの距離のことを言い昨年度の130 mmから管長を長くしていった.その結果,Fig.\ref{fig:intake2}のように管長が280 mm,タンク容量3.5 Lの時にピークトルクが7000 rpmとなり5000〜8000 rpmにかけての出力特性を向上することができた.今後は実測による評価を行う予定である.

%%%%%%%%%%%%%%%%%%%%%%%%%%%
\subsubsection{サージタンク}
%%%%%%%%%%%%%%%%%%%%%%%%%%%
サージタンクの容量はGT-powerの解析による結果から3.5 Lに決定し,形状は流体解析ソフトANSYS CFXを用いて各気筒への吸気流量にばらつきがないように形状を決定した.その結果,Fig.\ref{fig:intake3}のように各気筒への吸気流量のバラつき2.5 \%以下に抑えることができた.


%排気
%%%%%%%%%%%%%%%%%%%%%%%%%%%
\subsection{排気}
%%%%%%%%%%%%%%%%%%%%%%%%%%%
排気系設計ではパワートレインの目標を達成するため常用回転域での更なるトルク向上と扱いやすい出力特性を目指し開発を行った.目標として常用回転域での最大軸トルクを60 N-mとした(前年度45N-m).加えて,ドライバーが扱いやすいリニアなトルク特性となるようにエンジンモデルを選定した.

またKS-14が孕んでいた様々な問題についても解決策を模索した.

%%%%%%%%%%%%%%%%%%%%%%%%%%%
\subsubsection{排気管}
%%%%%%%%%%%%%%%%%%%%%%%%%%%
まずKS-14の問題点として解析方法の未確立及び前年度との比較不足が挙げられた.これは,9月の段階にはGT-powerでKS-14のエンジンモデルを製作して解析を行い、比較対象を明確化することで解決した.次にKS-15では排気管の軽量化にも取り組み,排気系パーツ一つ一つの重量を実測・管理を徹底した.またKS-14から採用したはめ込み式の排気管は取り付けがしづらかったため,はめ込み式は踏襲しつつ集合部のみを一体形状とし整備性の向上に努めた.集合部は左右対称形状とすることで排気管の取り回し設計や製作を行いやすくした.またA/Fセンサーマウントを各気筒に配置することで,気筒別に燃調マッピングが行える構造とした.(Fig.\ref{fig:power3})

次に軸トルク向上及び扱いやすい出力特性を達成するため,吸気側が先に製作した’17年度エンジンモデルを参考に,主に排気管長と管径を変更しながら10個ほどのモデルを作成した.これらの解析結果から,管径よりも管長,特に集合部よりも前の管長(プライマリ)が出力特性に大きく影響を及ぼすことが分かった.そこで集合部以下を燃料タンクやフレームとの干渉を考慮しつつ必要最小限の管長・管径とした.管径については解析結果から僅かな差であるが$\phi$38.1が最適であったため,それ以降の解析は管長のみを変更し管径は常に$\phi$38.1で行った.以上の方法で約40個のモデルを作成し,その中から目標最大トルクを達成しかつリニアなトルク特性となるものを選定した.その結果プライマリ管長は495 mmとなった.Fig.\ref{fig:power4}にKS-14の標準バルブタイミングのものとKS-15のそれぞれのトルク・PS性能曲線を示す.Fig.\ref{fig:power4}から分かるように常用回転域において大幅にトルクが向上したことが分かる.また最大トルクは7500 rpmにおいて60.1 N-mとなり目標数値を達成した.また最小限の管長・管径としたことにより排気管で200 gの軽量化を実現することが出来た.

%%%%%%%%%%%%%%%%%%%%%%%%%%%
\subsubsection{サイレンサ}
%%%%%%%%%%%%%%%%%%%%%%%%%%%
KS-14においては,アイドリング時の騒音低減を狙ってチャンバー(サブサイレンサ)を搭載していた.しかしチャンバー単体で1200 gもあることに加えスペースを必要とするためサイレンサを外側に配置しなければならず慣性モーメントの増加を招いていた.そこでKS-15ではまずチャンバー廃止によるエンジン出力への影響を検討した.Fig.\ref{fig:power5}がGT-powerで解析した結果であるが,出力への影響は殆どなかったため廃止をすることに決定した.しかし騒音については全域において音圧レベルが増加することが解析結果から分かったため,増加分をサイレンサエンド部を延長することにより消音することとした.

サイレンサエンド部の長さを10 mmずつ長さの違うモデルを複数個作成しGT-powerを用いて評価・選定を行った.その結果,サイレンサエンド部が550 mmのときアイドリング時2500 rpmのとき100.2 dBC,11000 rpmのとき108.7 dBCとなり必要な消音性能を得ることが出来た(Fig.\ref{fig:power6}).


%電装
%%%%%%%%%%%%%%%%%%%%%%%%%%%
\section{title}
%%%%%%%%%%%%%%%%%%%%%%%%%%%
aaaaaaaaaaaaaa

%%%%%%%%%%%%%%%%%%%%%%%%%%%
\subsection{subtitle}
%%%%%%%%%%%%%%%%%%%%%%%%%%%
bbbbbbbbbbbbbb

%%%%%%%%%%%%%%%%%%%%%%%%%%%
\subsubsection{subsubtitle}
%%%%%%%%%%%%%%%%%%%%%%%%%%%
cccccccccccccc


%冷却
%%%%%%%%%%%%%%%%%%%%%%%%%%%
\subsection{冷却}
%%%%%%%%%%%%%%%%%%%%%%%%%%%
Fig.\ref{fig:radiator1}にKS-14とKS-15のラジエータ配置を示す.KS-15の冷却システムは十分な冷却性能の確保を設計方針とした.昨年のエンデュランスでは水温が$110 \ {}^\circ\mathrm{C}$を超えておりこれはラジエータの配置によるものと考えられた.KS-14はラジエータの外側が後方に倒れるように搭載していたがコア部に空気がうまく流入せず空気側の熱伝達率が小さくなっていると考えられる.これに対しKS-15はコア部が正面を向くように配置しサイドポンツーンの兼ね合いで25$^\circ$前傾させた.これによって熱伝達率の向上と放熱量の向上が見込める.またラジエータを正面に向けたため従来のサイズでは車幅を超えるためサイズを縮小したところの軽量化となった.またステーの見直しにより0.43 kg軽量化できた.


%燃料
%%%%%%%%%%%%%%%%%%%%%%%%%%%
\subsection{燃料}
%%%%%%%%%%%%%%%%%%%%%%%%%%%
’18年度のチーム目標を達成するためにはエンデュランス完走が前提であり,またパワートレインの目標を達成するためには燃料系の総重量は6000gを切る必要があると考えた.そこで,燃料システムの目標を「安定した燃料供給と軽量化」とした.なお,この総重量は“’18 年度目標マシン重量“に“マシン全体に対する燃料系の重量割合“を掛けて求めたものである.

これを踏まえて,製作目標は「エンデュランス完走直後の燃料残量の状態でコーナーに侵入した際でもエア噛みが発生しないタンク」とし,解析ソフトANSYSから得られた解析結果からこれに見合うタンクの形状および容量を決定した.その際の解析条件を,“タンク内に0.7 Lの燃料残量がある状態で最大横G 1.8 G及び最大縦G 0.5 Gが発生する”とした.なお,エンデュランス完走直後の燃料残量0.7 Lは,タンクの容量5.0 L(’15年度大会のタンク容量が5 Lだったことからこれを指標とした)から燃料の使用量4.3 L(気温$20 \ {}^\circ\mathrm{C}$時の試走会で測定した燃費により決定)を引いたものである. 

上記の解析条件よりタンクの形状を決定したが,形状の工夫だけでは旋回時に発生する液体の偏りを抑えることが出来ず,一時的ではあるがエア噛みが発生する瞬間があった.そこで,液体の偏り及び挙動を抑えるために’18年度から構造体内にバッフルプレートを設けることにした(Fig.\ref{fig:fuel1}).その結果,燃料残量0.7 L時に旋回Gが発生した際の液体の挙動が小さくなり,エア噛みの発生を抑えることが出来た(Fig.\ref{fig:fuel2}).これにより,燃料タンクの容量を17年度の5.3 Lから5.0 Lへの変更が可能になり,ガソリンを含めたのタンクの重量を311 g減少することに成功した.また,’17年度のタンクに使用したネックパイプの素材がA5052の$\phi$45 mm,t5 mmであったため,’18年度は同素材の$\phi$41 mm,t2 mmのものに変更することで330 gの軽量化に成功した.さらに,タンクに使用するプレートに曲げ加工を施すことで,タンクの1面につきアルミ板を1枚ずつ切り出すよりも溶接長が647 mm短くなり,179 g(溶接長1 mmにつき0.277 g)の軽量化に成功した.

したがって,’17年度と比較すると全体で990 g(実測値)の軽量化に成功し,’18年度燃料系の総重量はガソリン満タン時で5500 gとなり目標重量を達成した.

