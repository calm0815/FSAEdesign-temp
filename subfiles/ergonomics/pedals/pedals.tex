%%%%%%%%%%%%%%%%%%%%%%%%%%%
\subsection{ペダルユニット}
%%%%%%%%%%%%%%%%%%%%%%%%%%%
%%問題点
昨年度は各ペダル共に足の指先部分での操作を強いられた.また,昨年度パーツを調査したところひび割れや歪みは見られず,踏み込む力に対し各パーツの剛性が過剰であったといえる.%% さらにペダルハウジングの複雑さから操作性と整備性の悪いものだった.

%%改善点
これらの問題点からレイアウトと部品設計の2つの変更を行った.

まず,コクピットの目標を達成するためペダルレイアウトの見直しを行った.モックアップを製作し,吊り下げ式とオルガン式を4人のドライバーに200回ずつ踏んでもらいフィードバックを得た.その結果, 吊り下げ式では4人ともすね側の筋肉が痛くなりオルガン式のほうが疲労度が少なかった.また,ペダルの各要素に重要度を設けAHP法を用い吊り下げ式とオルガン式を検討した.その結果,整備性とレイアウトの単純化の観点では吊り下げ式が良かったが,踏み込みやすさや重量,ノウハウの点でオルガン式の点数が高かったためオルガン式を採用した.

%%解析に関して
次にペダル長を昨年度比で10 mm短くし,ペダルの軽量化と足の母指球で踏み込める位置に設計した.剛性は昨年度ペダルが安全率1.5〜1.6で作られていたため安全率1.3を設計の目標値とした.軽量化と安全率を両立するため今年より構造最適化処理を設計に取り入れた.コンピュータにて構造最適化処理を施し,実現性の考慮を行った後,CATIAで強度解析を行った.その結果,前年度重量比でアクセル34 g減,クラッチ10 g減,マスターシリンダーステーは耐力を9 \%上げつつ74 g減量に成功した.ブレーキは剛性低下を考慮して大幅な減量は行わなかった.

%% 次にペダルレイアウトの見直しをした.踏み間違え等を考慮した結果,ペダル間隔は均一としワイヤステーの小型化を図った.これによりアクセルワイヤステーは昨年度比○mmペダルと間隔が広がり,○g,クラッチワイヤステーは○gの軽量化に成功した.さらに,ドライバーの要望により踏み込み部分に滑り止めを設け,ペダルの操作性を向上させた.
