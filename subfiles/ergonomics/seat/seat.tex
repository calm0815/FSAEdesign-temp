%%%%%%%%%%%%%%%%%%%%%%%%%%%
\subsection{シート}
%%%%%%%%%%%%%%%%%%%%%%%%%%%
\label{sec:seat}
今年度のコクピットの目標であるComfortabilityの向上とサスペンションの要求である低重心化を開発目標とした.%目指し設計・製作を行った.

KS-14では,パッケージングレイアウトでの連携不足によりシートの搭載スペースをコクピット内に十分確保することができず,背もたれの角度が20$^\circ$の極端なアップライトポジションとなり,長時間の耐久走行等でドライバーの体力に大きな負担をかける非常に窮屈なドライビングポジションとなっていた.そこで,KS-15では,フレームのモックアップを製作し,各ドライバーの快適なドライビングポジションを計測・検証した.
また,KS-15では新たにRr.ウィングを搭載するので,ドライバーの重心を低くすることで重心高の上昇を抑制することにした.その結果,背もたれの角度が35$^\circ$のリクライニングポジションとなり,ドライバーの重心高を91 mm下げることに成功した.

KS-14のシートでは,シートの外枠となる部分をFRPでスクエア状に作り,内部をウレタン・ゴムスポンジで埋めるようなシートを製作しようとしたが,コクピット内部のトレッドの関係からドライバーの体にフィットするように内部を埋めることができず,非常に座り心地の悪いものとなった.さらに,成人男性の95 \%を一つのシートではカバーすることができなかったため,二種類のシートを使うことにしたが,ドライバーの体格に合わせてシートを変更する必要があり整備性の悪化につながった.KS-15では,FRPで曲面を採用したシートを製作し,体をホールドするのに重要な部分(腰部,肩部等)にウレタンを使用した.加えて,エンデュランス競技を走るドライバーの要望により,長時間運転しても体に負担のかかりにくい肩部でも体を保持できるような形状を取り入れよりホールド性の高いものとした.

また,ドライバーによって体格は異なるため,低身長のドライバーにはシートの腰部・底部にクッションを挿入することで,Comfortabilityを確保した.

上記のシートの変更,またシートステーの締結方法を変更することで807 gの軽量化に成功した.
