%%%%%%%%%%%%%%%%%%%%%%%%%%%
\subsection{車両諸元\textcircled{\scriptsize1}(目標重量とトレッドとダウンフォース)}
%%%%%%%%%%%%%%%%%%%%%%%%%%%
まず,定常領域でタイムを短縮するための評価指標を考察した.タイムの短縮区間である実際のエンデュランスコースにおけるコーナーではコーナーの半径が一定でも車両は基本的にアウトインアウトのラインをとりドライバーやライン取りの違いから一定した評価を行うことは難しいと考えた.一方,定常領域では比較的ドライバーの差や条件が安定しやすいため,スキッドパッドを評価指標とすることが最適であると考えた.そこで目標とするエンデュランス順位とスキッドパッドの順位を比較するとある一定の相関関係がみられた(Fig.\ref{fig:sus1}).そこでチーム目標を達成できるスキッドパッドの目標タイムを算出すると5.21 sec(昨年比-0.13 sec)となった.

スキッドパッド5.21 secを達成するためのタイム短縮のパラメータとして車両重量 kg,前後トレッド mm,ダウンフォース(以下DF),タイヤ特性,前後重量比が挙げられる.その中からkg,mm,DFに着目しスキッドパッドにおけるタイムシミュレーションからそれぞれのパラメータにおいてどのくらいタイムを短縮できるか考察したところ(Fig.\ref{fig:sus2},Fig.\ref{fig:sus3},Fig.\ref{fig:sus4}),軽量化のタイムへの影響は0.003 sec/kg,DFにおけるタイムへの影響は0.0006 sec/N,トレッド短縮によるパイロン幅寄せではタイムは短縮できずコーナリングフォースの非線形性から逆にタイムは0.0003 sec/mmの割合で増加した.

前述の影響とオートクロス、エンデュランスのパイロンコースを勘案し,最大値のF=1200 [mm],R=1200 [mm]とした.次に,空力デバイス搭載におけるDF増とそれに伴う重量増のメリットデメリットを勘案したところ4.5 N/kg(得られるDF/空力デバイスの重量)以上のDFを発生できると空力デバイス搭載によるメリットが大きくなることがわかった.製作によって達成可能なエアロ重量を21 kg(Fr.ウイング:16 kg,サイドポンツーン:5 kg)とし,前述のタイムへの影響(DF/Mass(wing))を考慮してDF目標を輪荷重150 N@35-40 km/hとした.これにより約0.1 secの短縮が可能である.さらにウイング搭載における重量増分の軽量化を考慮したところ(Fig.\ref{fig:sus5})全体でこれを達成可能であると判断し,目標重量をウイング搭載で同等近くまで抑える(250 kg)ことにした.これにより約0.03 secの短縮が可能である.以上より目標タイム5.21 secを達成できる.

%%%%%%%%%%%%%%%%%%%%%%%%%%%
\subsection{車両諸元\textcircled{\scriptsize2}(前後重量比とホイールベース)}
%%%%%%%%%%%%%%%%%%%%%%%%%%%
前後重量比はスキッドパッドなどの定常領域だけを考慮すると50:50が最適であるが,エンデュランスコースは多くの複合コーナーと直線区間もあり,荷重移動量を考慮すると45:55が最適であると判断した.

ホイールベースは過渡領域であるパイロンスラローム区間においてタイム短縮を達成できる値とする.パイロンスラロームにおいてタイムを短縮するためにはヨー角加速度に注目すべきであると考えた.車が旋回を開始するメカニズムはドライバーがステアリングを操舵しFr.タイヤにスリップアングルが付きコーナリングフォース(以下CF)を発生しヨー角加速度の発生で車体がヨー運動を開始する.このヨー角加速度を大きくすると車両向き変えが速くなりパイロンスラロームのように車両の向き変えが多いセクションではヨー増加によるメリットは大きくなると考えた.ヨー角方向の運動方程式は次の式で表せる.
\begin{equation}
  \label{eq:undou}
  I_Z \ddot{\psi} = M_Z
\end{equation}
($I_Z$:慣性モーメント,$\ddot{\psi}$:ヨー角加速度,$M_Z$:タイヤが発生するモーメント)\\
式\ref{eq:undou}より次の式が得られる.
\begin{equation}
  \ddot{\psi} = \frac{M_Z}{I_Z}
\end{equation}
ヨー角加速度を大きくするためには$M_Z$を大きくするか,$I_Z$を小さくする必要がある.

KS-15はウィングの搭載が決まっており重心からホイールより遠い位置に重量の約7 \%を占める部品が配置されるためヨー角加速度が小さくなることが予測される.このヨー角加速度の減少をできるだけ小さく抑えることとした.そのために$M_Z$の増大(ホイールベースの延長)を行う.2016年度車両(以下KS-13)において1700 mmまで拡大してもオートクロスやエンデュランス走行におけるパイロンタッチのリスクはないことはわかっている.昨年度のオートクロスの順位からホイールベースとタイムの相関性を調べたところ(Fig.\ref{fig:sus6})ホイールベース延長によるデメリットは小さいと考え,今年度はホイールベースをベンチマークとするチームと同等程度の1720 mmとした.

%%%%%%%%%%%%%%%%%%%%%%%%%%%
\subsection{タイヤ選定}
%%%%%%%%%%%%%%%%%%%%%%%%%%%
タイヤ選定にあたってはホイール径10 inchと13 inchの2サイズを検討した.10インチホイールのメリットは軽量化である.10 inchと13 inchの重量を比較すると4輪で6.91 kgの軽量化が見込まれる.しかし,タイヤデータを比較するとCPやホイール内部の設計自由度減少の観点,さらに新規設計部品の増加による設計期間の延長が考えられる事から慣性モーメント低減やその他メリットは小さいと考え小さなスリップアングルで大きなCFを生み出す13 inch Hoosier20.5-7.0/13とした.

%%%%%%%%%%%%%%%%%%%%%%%%%%%
\subsection{サスペンションジオメトリ}
%%%%%%%%%%%%%%%%%%%%%%%%%%%
スキッドパッド目標タイム,またスラロームでの目標を達成する為に空力デバイスの搭載を考慮した設計を行った.また,設計期間短縮のためにジオメトリ目標値(主に,キャンバーチェンジ,トーチェンジ)を成績・ドライバーともに感触が良好であったKS-13と同等程度と設定し,小変更があっても目標範囲であればジオメトリ設計変更を行わない方針をとった.

%%%%%%%%%%%%%%%%%%%%%%%%%%%
\subsection{レイアウト}
%%%%%%%%%%%%%%%%%%%%%%%%%%%
KS-14ではFr.ダンパー周辺の軽量化を狙い車体サイドに配置していたが,ダンパー取り付け点がFr.フープブレースの結合点からほぼ中心の位置にあり,フレームが弾性変形していることで設計通りの動きが出来ていなったと考えられる.今年度はFr.ダンパーは従来通り製作性を考慮してFr.セクションの上部に配置した.また,リアは空力デバイスを搭載しRr.ウイングはばね下に搭載するためウイングロッド取り付け点の確保のため,またトーロッドにかかる力を減少させるためロアーアームにプッシュロッドを配置した.これによりプッシュロッドにかかる力を前年度比27.8 \%減少させ,アッパーアームのガゼット小型化にも貢献した.

%%%%%%%%%%%%%%%%%%%%%%%%%%%
\subsection{ジオメトリ変化}
%%%%%%%%%%%%%%%%%%%%%%%%%%%
前述の通り,ジオメトリ変化の目標値には幅を持たせた.今年度はFr.ウイングが搭載され,路面接触のリスクも高まることから,ロール剛性を前年度より高めに設定しロール量を抑えていることからジオメトリ変化も小さくなる.

キャンバー変化は旋回時に外輪が前後ともにポジティブにならないようキャンバー角を-1$^\circ$〜0$^\circ$の範囲(Fig.\ref{fig:susA})とし,トー変化はフロント0$^\circ$〜0.5$^\circ$(Fig.\ref{fig:susB}),リアは-0.5$^\circ$〜0$^\circ$とした.スカッフ変化はHeave25 mm時2 mm(Fig.\ref{fig:susC})とした.

%%%%%%%%%%%%%%%%%%%%%%%%%%%
\subsection{キングピン軸周りのジオメトリ}
%%%%%%%%%%%%%%%%%%%%%%%%%%%
KS-14と同様にキャスター角6.981$^\circ$,キングピン傾角13.097$^\circ$,キャスタートレール24.615 mmとした.

%%%%%%%%%%%%%%%%%%%%%%%%%%%
\subsection{サスペンションアーム}
%%%%%%%%%%%%%%%%%%%%%%%%%%%
サスペンションアームは従来通りスチール製パイプで製作し軽量化の為ロアーアームとアッパーアームで肉厚の異なるものを使用し軽量化を図った.さらに軽量化のためにアームの補強材に着目した.それぞれのアームにおいてガゼットの有無で解析を行ったところ安全率にほとんど変化はなかった.そこでガゼットを最小化しアーム類のみで-615 gの軽量化を行った.

%%%%%%%%%%%%%%%%%%%%%%%%%%%
\subsection{トーロッド}
%%%%%%%%%%%%%%%%%%%%%%%%%%%
KS-14のマシン特性にドライバーからのフィードバックでコーナリング限界が低くアンダーステア傾向がみられるというものがあった.この原因をトーロッドの剛性不足であると仮定した.トーロッドの剛性不足により旋回外輪がトーアウトになりコーナリング限界の低下につながっていると考え,$\phi$12アルミニウム中実棒製トーロッドと高剛性化を狙った$\phi$15.9のスチールパイプ製のトーロッドを比較し今年度はスチール製を採用した.

%%%%%%%%%%%%%%%%%%%%%%%%%%%
\subsection{タイロッド}
%%%%%%%%%%%%%%%%%%%%%%%%%%%
タイロッドは昨年度の後ろ引きから,コンプライアンストーアウトを目的に前引きに変更しトーロッドと同様にタイロッドをスチールパイプに変更しステアリングフィールの向上を目指した.

%%%%%%%%%%%%%%%%%%%%%%%%%%%
\subsection{バネライン系}
%%%%%%%%%%%%%%%%%%%%%%%%%%%
バネライン系の詳細を決定するにあたってまずは最低地上高,目標ロール角を設定した.KS-14はインリフトの抑制のためにフロントのホイールレートを高くしていたがマシンのアンダーステア傾向が大きくなってしまった.また,最低地上高は低いほうが重心高が下がる利点がある一方,走行中の路面タッチのリスクも高まる.最低地上高管理とRr.ウイング搭載による重心高の上昇の相殺を含め,今年度はホイールレートを見直し,最低地上高を30 mm(20 mm+10 mm)に設定し,ホイールストローク量を20 mm/G(最大荷重2 G)に設定した.バネの限界ストローク量は76.2mmであり,ライドクリアランスを20 mmとるとバネストローク量は56.2 mmとなる.ホイールストローク量からホイールレートは決まりフロント38.1 N/mm,リア46.6 N/mmとなる.バネストローク量からモーションレシオは前後とも0.676に設定した.スプリングレートはフロント17.4 N/mmリア21.3 N/mmとした.空力デバイスが装着されたマシンの想定重量はドライバー込みで310 kg(ドライバー60 kg+車体重量250 kg)より,バネ固有振動数はフロント3.96 Hzリア3.99 Hzとなった.(KS-14 Fr:3.90 Hz,Rr:3.43 Hz)

%%%%%%%%%%%%%%%%%%%%%%%%%%%
\subsection{重心高}
%%%%%%%%%%%%%%%%%%%%%%%%%%%
空力デバイスの搭載により,主にRr.ウイングによって重心高が上がることが予想される.この重心高の上昇は荷重移動量の増加からコーナリング性能の低下を招く.スキッドパッドでのタイムの影響は(Fig.\ref{fig:susD})の様に0.002 sec/mmの割合で影響を及ぼす.この重心高の上昇をドライバーの運転姿勢をリクラインにすることで抑制することにした.詳細は\ref{sec:seat}項で記述する.
